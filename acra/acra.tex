%%%% acra.tex

\typeout{ACRA Instructions for Authors}

% This is the instructions for authors for ACRA.
\documentclass{article}
\usepackage{acra}
% The file acra.sty is the style file for ACRA. 
% The file named.sty contains macros for named citations as produced 
% by named.bst.

% The preparation of these files was supported by Schlumberger Palo Alto
% Research, AT\&T Bell Laboratories, and Morgan Kaufmann Publishers.
% Shirley Jowell, of Morgan Kaufmann Publishers, and Peter F.
% Patel-Schneider, of AT\&T Bell Laboratories collaborated on their
% preparation. 

% These instructions can be modified and used in other conferences as long
% as credit to the authors and supporting agencies is retained, this notice
% is not changed, and further modification or reuse is not restricted.
% Neither Shirley Jowell nor Peter F. Patel-Schneider can be listed as
% contacts for providing assistance without their prior permission.

% To use for other conferences, change references to files and the
% conference appropriate and use other authors, contacts, publishers, and
% organizations.
% Also change the deadline and address for returning papers and the length and
% page charge instructions.
% Put where the files are available in the appropriate places.

\title{ACRA Format Instructions for Authors}
\author{FirstName LastName \\ ACME University, Australia \\ 
emailaddress}

\begin{document}

\maketitle

\begin{abstract}
The {\it ACRA Proceedings} will appear in CD-ROM form only.
To ensure that all papers in the {\it Proceedings} have a
uniform appearance, authors are asked to adhere to the following
instructions. In addition, we will accept, and in fact encourage, 
submissions in the final format. This file includes the style instructions 
for submissions.
\end{abstract}

\section{Introduction}

Although ACRA is a CD-ROM only conference, papers should be prepared so that
they can be printed out. 

\subsection{Corresponding Author Details} 
The corresponding author is requested to email the following
information along with the paper: 1. title of the paper, 
2. name and postal address, email address.

\subsection{Word Processing Software}

As detailed below, ACRA has prepared and made available a set of
\LaTeX{} macros and Word templates for use in formatting your paper.
If you are using some other word processing software (such as
WordPerfect, etc.), please follow the format instructions given below
and ensure that your final paper looks as much like this sample as
possible.

\section{Style and Format}

\LaTeX{} and Bib\TeX{} style files, and Word templates that implement these 
instructions can be retrieved electronically.  See the ACRA homepage for 
details under
\verb+http://www.araa.asn.au/acra+

\subsection{Layout}

Prepare manuscripts two columns to a page, in the manner in which these
instructions are printed.  The exact dimensions for pages are:
\begin{itemize}
\item left and right margins: $.75''$
\item column width: $3.375''$
\item gap between columns: $.25''$
\item top margin---first page: $1.375''$
\item top margin---other pages: $.75''$
\item bottom margin: $1.25''$
\item column height---first page: $6.625''$
\item column height---other pages: $9''$
\end{itemize}

All measurements assume an {\bf $8$-$1/2 \times 11''$} page size.  
For A4-size paper use the given top and left margins, column width,
height, and gap and modify the bottom and right margins as necessary.

\subsection{Title and Author Information}

Center the title on the entire width of the page in a 14-point bold font.
Place the names of authors below the title in a 12-point bold font, and
affiliations and complete addresses directly below the author names in a
12-point (non-bold) font.

Credit to a sponsoring agency appears in a footnote at the bottom of the
left column of the first page.  See the example in these instructions.

\subsection{Abstract}

Place the abstract at the beginning of the first column $3.0''$ from the
top of the page, unless that does not leave enough room for the title and
author information.  Use a slightly smaller width than in the body of the
paper.  Head the abstract with ``Abstract'' centered above the body of the
abstract in a 12-point bold font.  The body of the abstract should be in
the same font as the body of the paper.

The abstract should be a concise, one-paragraph summary
describing the general thesis and conclusion of your
paper. A reader should be able to learn the purpose of the
paper and the reason for its importance from the abstract. The
abstract should be no more than 200 words long.

\subsection{Text}

The main body of the text immediately follows the abstract. 
Use 10-point type in a clear, readable font with 1-point leading (10 on
11).  For reasons of uniformity, use Computer Modern font if possible.  If
Computer Modern is unavailable, Times Roman is preferred.

Indent when starting a new paragraph, except after major headings.

\subsection{Headings and Sections}

When necessary, headings should be used to separate major sections of your
paper.
(These instructions use many headings to demonstrate their
appearance---your paper should have fewer headings.)

\subsubsection{Section Headings}

Print section headings in 12-point bold type in the style shown in these
instructions.  Leave a blank space of approximately 10 points above and 4
points below section headings.  Number sections with arabic numerals.

\subsubsection{Subsection Headings}

Print subsection headings in 11-point bold type.  Leave a blank space of
approximately 8 points above and 3 points below subsection headings.
Number subsections with the section number and the subsection number (in
arabic numerals) separated by a period.

\subsubsection{Subsubsection Headings}

Print subsubsection headings in 10-point bold type.  Leave a blank space of
approximately 6 points above subsubsection headings.  Do not number
subsubsections.

\subsubsection{Special Sections}
The acknowledgments section, if included, follows the main body of the text
and is headed ``Acknowledgments,'' printed in the same style as a section
heading, but without a number. 
This section includes acknowledgments of help from
colleagues, financial support, and permission to publish.  
Please try to limit acknowledgments to no more than three sentences.

Any appendices follow the acknowledgments (or directly follow the text) and
look like sections, except that they are numbered with capital letters
instead of arabic numerals. 

The references section is headed ``References,'' printed in the same
style as a section heading, but without a number.
A sample list of references is given at the end of these
instructions.
Use a consistent format for references, such as provided by
Bib\TeX{}.

\subsection{Citations}

Citations within the text should include the author's last name and
the year of publication, for example \cite{cheeseman:probability}.
Append lowercase letters to the year in cases of ambiguity.
Treat multiple authors as in the following examples:
\cite{abelson-et-al:scheme} (for more than two authors) and
\cite{brachman-schmolze:kl-one} (for two authors).
If the author portion of a citation is obvious, omit it,
e.g., Levesque \shortcite{levesque:belief}.
Collapse multiple citations as follows:
\cite{levesque:functional-foundations,haugeland:mind-design}.%
\nocite{abelson-et-al:scheme}%
\nocite{brachman-schmolze:kl-one}%
\nocite{cheeseman:probability}%
\nocite{haugeland:mind-design}%
\nocite{lenat:heuristics}%
\nocite{levesque:functional-foundations}%
\nocite{levesque:belief}

\subsection{Footnotes} 
Place footnotes at the bottom of the page in a 9-point font.\footnote{This is how your footnote should appear.} 
Refer to them with superscript numbers. 
Separate them from the text by a short line.\footnote{Note the line seperating these footnotes from the text.}
Avoid footnotes as much as possible; they interrupt the flow of the text.

\section{Illustrations}

\subsection{General Instructions}
Place illustrations (figures, drawings, tables, and photographs) throughout the paper at the places where they are first discussed, rather than at the end of the paper. If placed at the bottom or top of a page, illustrations may run across
both columns. Securely attach them to the master form with glue stick, spray 
adhesive, rubber cement, or white tape. Do not use transparent tape as the 
printing process blurs copy under transparent tape.

Number illustrations sequentially. Use references of the following form: 
Figure 1, Table 2, etc. Place illustration numbers and captions under 
illustrations. Leave a margin of 1/4-inch around the area covered by the 
illustration and caption. Use 9-point type for captions, labels, and other 
text in illustrations.

1This is how your footnotes should appear. 2Note the line separating these footnotes from the text.

Do not use line-printer printouts or screen-dumps for figures--they will be illegible when printed. Avoid screens or pattern fills as they tend to reproduce poorly.

\section{Length of Papers} 
Submissions must should be within 6 to 10 pages in length. 

\section*{Acknowledgments}
The preparation of these instructions and the LaTEX and BibTEX files that implement them was supported by Schlumberger Palo Alto Research, AT\&T Bell Laboratories, and Morgan Kaufmann Publishers.

\section*{Format Files}
Using LaTEX A LaTEX style file for version 2.09 of LaTEX that implements these instructions has been prepared, as has a BibTEX style file for version 0.99c of BibTEX (not version 0.98i) that implements the citation and reference styles here.

There is also a Word 6.0 template available in RTF-format.

The relevant files are available from the ARAA web server.

\begin{center}
\begin{verbatim}
http://www.araa.asn.au/acra
\end{verbatim}
\end{center}

As the files may be changed to fix bugs, you should ensure that you are using the most recent versions.

%% This section was initially prepared using BibTeX.  The .bbl file was
%% placed here later
%\bibliography{publications}
%\bibliographystyle{named}
%% The file named.bst is a bibliography style file for BibTeX 0.99c
\begin{thebibliography}{}

\bibitem[\protect\citeauthoryear{Abelson \bgroup \em et al.\egroup
  }{1985}]{abelson-et-al:scheme}
Harold Abelson, Gerald~Jay Sussman, and Julie Sussman.
\newblock {\em Structure and Interpretation of Computer Programs}.
\newblock MIT Press, Cambridge, Massachusetts, 1985.

\bibitem[\protect\citeauthoryear{Brachman and
  Schmolze}{1985}]{brachman-schmolze:kl-one}
Ronald~J. Brachman and James~G. Schmolze.
\newblock An overview of the {KL-ONE} knowledge representation system.
\newblock {\em Cognitive Science}, 9(2):171--216, April--June 1985.

\bibitem[\protect\citeauthoryear{Cheeseman}{1985}]{cheeseman:probability}
Peter Cheeseman.
\newblock In defence of probability.
\newblock In {\em Proceedings of the Ninth International Joint Conference on
  Artificial Intelligence}, pages 1002--1009, Los Angeles, California, August
  1985. International Joint Committee on Artificial Intelligence.

\bibitem[\protect\citeauthoryear{Haugeland}{1981}]{haugeland:mind-design}
John Haugeland, editor.
\newblock {\em Mind Design}.
\newblock Bradford Books, Montgomery, Vermont, 1981.

\bibitem[\protect\citeauthoryear{Lenat}{1981}]{lenat:heuristics}
Douglas~B. Lenat.
\newblock The nature of heuristics.
\newblock Technical Report CIS-12 (SSL-81-1), Xerox Palo Alto Research Centers,
  April 1981.

\bibitem[\protect\citeauthoryear{Levesque}{1984a}]{levesque:functional-foundat%
ions}
Hector~J. Levesque.
\newblock Foundations of a functional approach to knowledge representation.
\newblock {\em Artificial Intelligence}, 23(2):155--212, July 1984.

\bibitem[\protect\citeauthoryear{Levesque}{1984b}]{levesque:belief}
Hector~J. Levesque.
\newblock A logic of implicit and explicit belief.
\newblock In {\em Proceedings of the Fourth National Conference on Artificial
  Intelligence}, pages 198--202, Austin, Texas, August 1984. American
  Association for Artificial Intelligence.

\end{thebibliography}

\end{document}

