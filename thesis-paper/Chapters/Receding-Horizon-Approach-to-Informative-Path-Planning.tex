\lhead{Receding Horizon Approach to Informative Path Planning}
\chapter{Receding Horizon Approach to Informative Path Planning}
\label{RecedingHorizonApproach}

	\section{Motivation}
	
		Talk about MPC and its use in control theory
		
	\section{Development of Method}
	
%		\subsection{Theory}
%		
%		\subsection{Implementation}
		
		Provide a flow diagram?
		
	\section{Properties of a Receding Horizon Solution}
	
		\subsection{Horizon Length}
		
		\subsection{Step Spacing}
		
		\subsection{Path Generation \& Natural Coordinates}
		
		\subsection{Turn Angle Limits for Smooth Paths}
		
		\subsection{Feature Space Transformations}
		
	\section{Computational Aspects}
	
		\subsection{Optimisation Process and Bottlenecks}
		
		\subsection{Relearning the Environment}
		
		\subsection{Linearised Entropy Approach}
		
		\subsection{Monte Carlo Approach}
		
	\section{Feasibility and Practicality}
	
	\section{Results with Abundant Uniform Test Data Set}
	
	\section{Results with Scarce Uniform Test Data Set}
	
	\section{Results with Simulated Track Data}
	
	\section{Results with Scott Reef Data}
	
		\subsection{Ground Truth Generation}
		
		\subsection{Practical Considerations}
		
		\subsection{Results}
		
	\section{Performance Assessment}
	
	
	