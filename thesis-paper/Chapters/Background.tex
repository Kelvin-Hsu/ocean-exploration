\lhead{Background}
\chapter{Background}
\label{Background}

	\section{Active Sampling}
	
	\section{Informative Path Planning}
	\label{Background:PathPlanning}
	
		Path planning under dynamic uncertainty has been a challenging task for all information searching missions. This class of path planning problems have the special property that there is no goal location, and no stationary node, edge, or field cost to be cumulatively minimised. The objective is to reduce the overall uncertainty or entropy of a particular region given indefinite time. The complication is introduced with the non-linear dynamics of the uncertainty or entropy field of the region each time a planned path is executed, which also makes the solution extremely frequency dependent.
		
		Prior work and attempts at the active path planning problem include Marchant and Ramos (2014), where Bayesian Optimisation (BO) techniques combined with Gaussian process models are employed in an environmental monitoring setting \cite{BayesianOptimisation}. In this layered Bayesian Optimisation approach, two Gaussian processes are used - one to model the phenomenon and the other to model the quality of selected paths. Through Bayesian optimisation, sampling over continuous paths are optimised which maximises the reward over the final mission trajectory. The path planning process is done using Markov Decision Processes (MDP) with a Reinforcement Learning approach. Rapidly Exploring Random Graphs (RRGs) is combined with BO to search for informative paths. In this way, a continuous path can be planned through BO \cite{BayesianOptimisation}.
		
		This was was further extended by Marchant et. al. (2014) where Sequential Bayesian Optimisation (SBO) is used as online POMDPs for path planning \cite{SequentialBayesianOptimisation}.
		
		Other prior work includes Brooks et. al. (2006) where the POMDP approach is investigated for continuous state space planning \cite{ParametricPOMDP}. This method was compared to previous work with value-based and gradient-based solution methods which seek to transcribe the continuous problem into a discrete problem \footnote{{\color{BurntOrange} The author has practiced with transcribing the continuous problem into a discrete problem in a shortest path setting in}}. One of the most important limitations discussed in this work is that analytical and accurate solutions exist almost only for linear systems with quadratic cost (Linear Quadratic Systems). Otherwise, the other option with non-value based methods require heuristics that can be difficult to justify for its appropriateness to the problem. Nevertheless, through parametrising the problem, parametric POMDPs can provide an accurate solution to the path planning problem under certain assumption such as linear quadratic dynamics \cite{ParametricPOMDP}.
		
		In conclusion, POMDP methods are currently one of the most reliable and accurate method for continuous path planning in an information gathering setting. While the technique enforces limitations on the problem dynamics, with sufficient modeling it is deemed possible to perform path planning to an acceptable level. This thesis will thus investigate POMDP methods for active path planning in the ocean exploration setting.
		
		\subsection{Myopic and Nonmyopic Planning}
		
		\subsection{Advantages of Gaussian Process Models}