\chapter{Introduction}
\lhead{Introduction}
\label{Introduction}

	\section{Motivation}
	\label{Introduction:Motivation}
	
		Thanks to optical and acoustic depth sounding technology, detailed ocean terrain maps across a majority of the globe have become increasingly accessible. These information generally takes the form of \textit{Bathymetric} data - recordings of measured depth, slope, roughness, and similar structural information that summarises the seafloor topography. Currently, bathymetric data has been recorded with advanced techniques such as SONAR (\textbf{SO}und \textbf{N}avigation \textbf{A}nd \textbf{R}anging), LIDAR (\textbf{LI}ght \textbf{D}etection \textbf{A}nd \textbf{R}anging), and Multibeam Echosounder for more than half a century \citep{Niedzielski2013231, Colbo201441}. With such volume of bathymetric information, we can reconstruct accurate 3D models for the seafloor terrain through spatial analytics and modeling techniques \citep{Niedzielski2013231}.
		
		However, bathymetric data only contains information regarding the spatial structure of the marine terrain. It provides no indication towards the types of marine habitats that resides within parts of the ocean, nor does it contain clues regarding the minerals or natural resources that may be present. Today, less than five percent of seafloor habitats have been explored \citep{NOAA}. As the ocean covers more than 70\% of the globe, this leaves more than 67\% of the planet's habitats unexplored despite our deep reliance on much of these undiscovered ecosystems. With big data analysis becoming more feasible in recent years, there has been an increase in scientific and economical demands - from ecologist and geologists to resource and mining industries - for the ability to predict or infer the types of marine habitats or natural resources residing at various marine environments.
		
		Thus, unlike the case for bathymetric data, there is currently a lack of \textit{label} data, which is a summary of the habitats, resources, and other interesting properties observed at various parts of the ocean. This implies the need to map the ocean floor again for label data using vision based sensing equipments. In order to understand the ecological, geological, chemical, and archaeological aspects of the ocean floor, autonomous underwater vehicles (AUVs) are now capable of efficiently collecting information and observations from natural environments of large spatial scale. In the case of benthic habitat mapping, AUVs collect imagery data of seafloor environments, which are then associated with a particular \textit{label} with semantic meaning \citep{Steinberg2015128}. For example, imageries of `coral' regions receive the label `coral'. Unfortunately, while bathymetric data can often be measured with decent accuracy at a distance (for example, with SONAR from ships at sea level), such visual imagery can only be obtained through expensive AUV missions that travel deep into the ocean to image underwater environments at a close distance. Together with the immense spatial scale of the benthic seafloor to be explored, this implies that it is impractical to map exhaustively the entire region of interest (ROI) under any reasonable time and cost. Furthermore, AUV missions are limited by power supply, data storage, and computational capabilities \citep{AsherBender}, further limiting the time and hence coverage each AUV mission can achieve. 
		
		As such, AUVs must prioritise exploring sub-domains of the ROI that ideally contain the most important and valuable information. This is a form of spatial sampling problem \citep{Rigby:ROB20372}, which aims to address the question: given the choice to observe only a few parts of the region of interest, how should one infer the best candidates for observation? AUV missions add another layer of complication to the spatial sampling problem - the candidate locations must form continuous paths that the AUV can physically travel.
		
		This is known as the \textit{informative path planning problem}. The objective of informative path planning is to minimise the overall uncertainty regarding the entire region of interest by traversing the most informative path.
		
		This thesis addresses the informative path planning problem for benthic habitat mapping. There are two main aspects to informative seafloor exploration for which this thesis is concerned with. The first part of this thesis is focused on benthic habitat mapping, where techniques of habitat classification and inference are discussed. The basic theory under which information and uncertainty are measured and quantified are developed and formulated in the general framework, which is then applied to benthic habitat mapping. The second part of this thesis then proceeds to investigate the informative seafloor exploration problem. Using the properties of inference models developed for benthic habitat mapping, a range of path planning policies are discussed and compared. Practical considerations of computational tractability and flexibility then lead to compromises between optimality and feasibility. Finally, this thesis proposes a practical framework for AUVs to autonomously plan informative paths that achieves the highest mapping rate under a classification accuracy criterion.
		
	\section{Objectives}
	\label{Introduction:Objective}
		The high level objective of this thesis is to develop an informative seafloor exploration policy that can produce benthic habitat maps efficiently. Specifically, this thesis focuses on the theoretical and computational aspects of informative path planning that is practical for seafloor mapping. The aim is to address the informative seafloor exploration problem in a principled manner with theoretical grounding, while taking computational feasibility into account. 
		
		The goal of this thesis can be summarised as follows:
		
		\begin{quote}
			To investigate informative seafloor exploration policies for an AUV with limited mission time, in order to map benthic habitats faster in a principled and computationally feasible way.
		\end{quote}
%	
%		The high level objective of this thesis is to develop and design an underwater path planner such that the resulting path minimises the overall mapping error of the seafloor region of interest.
%		
%		Detailed examination of this objective would raise details that would need to be made more specific.
%		
%		Firstly, the measure of entropy would be highly dependent on the quantities or qualities the vehicle is to search for, as well as the underwater region of interest. It would need to be examined to ensure that it is an appropriate measure of the environment uncertainty to be reduced that is relevant to the mission.
%		
%		Secondly, finding paths that subsequently minimises the overall entropy is fundamentally a dynamic programming and optimisation problem. As with any optimisation problem, problem constraints are to be defined and made clear. Within the presence of possibly difficult dynamical constraints, it is likely that simplifications are necessary at various stages of the thesis.
%		
%		Another consideration is the feasibility of the algorithm. While it can be difficult to measure the optimality of any solution proposed, it is often easier to examine the feasibility of the algorithm through studying the hardware constraints involved or the physical environment. One of the most important feasibility constraint relevant to this thesis is the computational capabilities of the vehicle computer hardware. Depending on the final proposed algorithm, it may not always be possible for the path planer to be executed in a fully online fashion. A more likely situation would involve planning a path to be executed for a duration of time and update the path through re-planning at a frequency much lower than the vehicle is navigating. For short missions, it may be more desirable to have the entire path planned offline.
%		
%		Other subtleties arise from the nature of the the underwater path planning problem. There is often no goal location, only an objective to maximise the information gained. This is a form of active path planning, which is similar to the active sampling philosophy but with dynamic constraints.
%		
%		Finally, other than the path planning algorithm, a major part of this thesis revolves around modeling the ocean environment accurately. Without an accurate understanding of the ocean environment, the vehicle cannot plan a path that is of reasonable significance. This thesis is to address the ways the ocean environment is to be modeled, develop these algorithms, and implement them on an ocean exploration setting.
%		
%		The detailed objective of this thesis is thus to examine and address all the concerns above, and to provide a better understanding towards the methods these active path planning problems can be approached.
		
	\section{Contribution}
	\label{Introduction:Contribution}
	
		This thesis is concerned with mapping seafloor benthic habitats efficiently in a principled manner. Specific contributions and focuses of this thesis are:
		
		\begin{itemize}
		
			\item A computationally efficient, parallelisable implementation for multiclass Gaussian process (GP) classifiers through the One versus All (OVA) scheme. Time and spatial complexity of the classifier are further improved through taking advantage of the GP classifier structure. Under Laplace approximation, this approach avoids the computational complexity from Monte Carlo sampling that is required in the inference stage. The computational framework is consistent with sparsification techniques for further complexity improvement although this is not necessary.
			
			\item An All versus All (AVA) approach to GP multiclass classification. While OVA techniques have been employed in various settings in the literature, there is no direct treatment of GP multiclass classification using All versus All classifiers, which achieves predictions with less biases due to a more balanced binary comparison between classes \textit{under balanced data}. It also shares the above advantages with OVA classifiers. This provides another choice for GP multiclass classification apart from the OVA approach.

			\item Two new ways for probability fusion in the GP multiclass classification setting under the One versus All and All versus All approach - mode keeping and exclusion. For the All versus All case, techniques for pre-fusing the probabilities into an appropriate form is also investigated and derived. These methods provide better properties when used for compute entropy of predictions as compared to simple normalisation techniques.
			
			\item A Monte Carlo based method for estimating the joint prediction information entropy hereby named \textit{Monte Carlo Prediction Information Entropy} (MCPIE). This work aims to provide a framework to obtain the joint prediction information entropy that is traditionally missing the the treatment of Gaussian process classifiers. While there are no analytical forms for the joint prediction information entropy available, it can be shown that the Monte Carlo estimation converges in value under reasonable amounts of sample draws. A stable and efficient method for Monte Carlo Optimisation is also devised which allows MCPIE to be utilised with reasonable tractability in an information acquisition setting, hereby named \textit{MCPIE acquisition}.
			
			\item An alternative, analytically tractable entropy measure hereby named \textit{linearised model differential entropy} (LMDE). The LMDE of GP classifiers captures mutual information of a given region of interest. Specifically, instead of examining uncertainty with mainly prediction variance, LMDE also takes into account the prediction bias. Through addressing the bias-variance trade-off, a common challenge in machine learning algorithms, LMDE provides an alternative way to quantify uncertainty and information. Analytical tractability is then achieved through linearisation approximations. Linearised model differential entropy is proposed as an acquisition function for informative path planning, hereby referred to as \textit{LMDE acquisition}.
			
			\item A receding horizon framework to informative path planning. This provides a framework that is computationally efficient yet produce informative paths that are stable in performance. The main advantage of this approach is its low computational requirement in time and memory as it is not necessary to compute the acquisition criterion across the entire ROI as done in many informative seafloor exploration policies. Together with LMDE and MCPIE acquisition, the suitability of the receding horizon approach is demonstrated on the Scott Reef data set from \cite{IMOS}. This thesis demonstrates that under a receding horizon framework, LMDE and MCPIE acquisition achieves a faster mapping rate than other non-mutual, myopic, or non-informative approaches, and does so with reasonable computational resources. 
			
		\end{itemize}
			
	\section{Structure}
	\label{Introduction:Structure}
	
		The structure of this thesis is outlined as below.
		
%		Chapter \ref{Background} provides the necessary background and theory for the purpose of understanding this thesis. Related work in informative seafloor exploration are presented, as well as the various approaches that has been undertaken in this area. Most importantly, this chapter introduces and summarises the Gaussian process framework.
%		
%		Chapter \ref{BenthicHabitatMapping} details how the benthic habitat environment are modeled upon bathymetric features using Gaussian process classifiers. Most importantly, this chapter extends the usual Gaussian process classifier framework to multiclass classification that is not limited to Laplace approximation only. A general OVA and AVA framework for multiclass GP classification is developed, which is then applied to synthetic and real datasets to verify performance. These frameworks demand the use of probability fusion techniques in order to produce consistent inference and prediction results. The chapter concludes by applying the GP models developed to map the benthic habitat mapping at Scott Reef.
%		
%		Chapter \ref{InformativeSeafloorExploration} proceeds to focus on the theory and applications of informative path planning. The outset of this chapter begins by formulating a Monte Carlo approach for estimating the joint prediction information entropy of a prediction. The highlight of this chapter is the introduction and derivation of the linearised model differential entropy (LMDE) for both binary and multiclass GP classifiers. Comparisons with the usual prediction information entropy is presented to demonstrate their differences and respective advantages and disadvantages. This motivates the use of linearised model differential entropy as a suitable acquisition function for informative path planning. The discussion continues with a formulation of the receding horizon approach to informative path planning, and moves on to demonstrate the application of various acquisition criteria under this formulation. Description of the simulation experiment are provided, and properties of the approach observed from corresponding results are discussed and analysed. The performance of the proposed exploration policy are then assessed with a classification accuracy criterion.

		Chapter \ref{Background} provides the necessary background and theory for the purpose of understanding this thesis. After an introduction to the theory behind Bayesian modeling with Gaussian processes (section \ref{Background:GaussianProcesses}), a review of current literature in informative path planning quickly reveals the need for a more computationally efficient approach that is non-myopic while capturing mutual information (section \ref{Background:RelatedWork}).
		
		Chapter \ref{BenthicHabitatMapping} then begins by building upon the existing Gaussian process classification framework for benthic habitat mapping. The mapping problem is first formulated as a supervised learning problem which requires the habitat labels to be modeled upon appropriate bathymetric features (section \ref{BenthicHabitatMapping:BathymetricFeatures}). Through examining the binary classification framework, efficient and parallelisable multiclass classifiers in the OVA and AVA scheme are developed in a way that is not limited to Laplace approximation. These frameworks demand the use of probability fusion techniques in order to ensure consistent inference. As such, on top of simple normalisation, two novel probability fusion methods, mode keeping and exclusion, are devised which exhibit improved inference properties as demonstrated by illustrative test data (section \ref{BenthicHabitatMapping:Classification}). This chapter concludes by applying the GP classifiers developed to map the benthic habitats at Scott Reef, which sets up the initial scenario for informative exploration experiments (section \ref{BenthicHabitatMapping:ScottReef}).
		
		Chapter \ref{InformativeSeafloorExploration} moves on to investigate informative path planning techniques that build upon such Gaussian process classifiers. Through motivating the need for a measure of mutual information, a direct approach with Monte Carlo estimation is formulated in order to compute the \textit{joint} prediction information entropy, hereby named Monte Carlo prediction information entropy (MCPIE) (section \ref{InformativeSeafloorExploration:MCPIE}). Careful analysis of the predictive probabilities obtained under likelihood response transformations of the latent function then inspires the concept of model differential entropy (MDE). Analytical tractability is then achieved through a linearisation technique for the binary, multiclass OVA, and multiclass AVA scenario, thus producing the novel linearised model differential entropy (LMDE) mutual information measure (section \ref{InformativeSeafloorExploration:LMDE}). The two measures are compared in properties and performance, revealing their complementary nature which resembles the bias-variance trade-off that is ubiquitous in statistics and machine learning (section \ref{InformativeSeafloorExploration:ComparisonMutualEntropyMeasures}). A stable and efficient path planning scheme is then developed under the receding horizon formulation (section \ref{InformativeSeafloorExploration:RecedingHorizonFormulation}), where the implementation details that improve its tractability are outlined. The discussion proceeds to demonstrate the application of various acquisition criteria under this formulation. Description of the simulation experiment are provided, and properties of the approach observed from corresponding results are discussed and analysed. The performance of the proposed exploration policy are then assessed with a classification accuracy criterion, which reveals the effectiveness of MCPIE and especially LMDE acquisition approaches over alternative methods (section \ref{InformativeSeafloorExploration:ScottReef}).
				
		Finally, Chapter \ref{Conclusion} summarises the work presented in this thesis, the contributions made, and potentials for future improvements.