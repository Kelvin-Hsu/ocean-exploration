\chapter{Introduction}
\lhead{Introduction}
\label{Introduction}

	\section{Motivation}
	
		Thanks to optical and acoustic depth sounding technology, detailed ocean terrain maps across a majority of the globe have become increasingly accessible. These information generally takes the form of \textit{Bathymetric} data - recordings of measured depth, slope, rugosity, and similar structural information that summarises the seafloor topography. Currently, bathymetric data has been recorded with advanced techniques such as SONAR (\textbf{SO}und \textbf{N}avigation \textbf{A}nd \textbf{R}anging), LIDAR (\textbf{LI}ght \textbf{D}etection \textbf{A}nd \textbf{R}anging), and Multibeam Echosounder for more than half a century \citep{Niedzielski2013231, Colbo201441}. With such volume of bathymetric information, we can reconstruct accurate 3D models for the seafloor terrain through spatial analytics and modeling techniques \citep{Niedzielski2013231}.
		
		However, bathymetric data only contains information regarding the spatial structure of the marine terrain. It provides no indication towards the types of marine habitats that resides within parts of the ocean, nor does it contain clues regarding the minerals or natural resources that may be present. Today, less than five percent of seafloor habitats have been explored \citep{NOAA}. As the ocean covers more than 70\% of the globe, this leaves more than 67\% of the planet's habitats unexplored despite our deep reliance on much of these undiscovered ecosystems. With big data analysis becoming more feasible in recent years, there has been an increase in scientific and economical demands - from ecologist and geologists to resource and mining industries - for the ability to predict or infer the types of marine habitats or natural resources residing at various marine environments.
		
		Thus, unlike the case for bathymetric data, there is currently a lack of \textit{label} data, which is a summary of the habitats, resources, and other interesting properties observed at various parts of the ocean. This implies the need to map the ocean floor again for label data using vision based sensing equipments. In order to understand the ecological, geological, chemical, and archaeological aspects of the ocean floor, autonomous underwater vehicles (AUVs) are now capable of efficiently collecting information and observations from natural environments of large spatial scale. In the case of benthic habitat mapping, AUVs collect imagery data of seafloor environments, which are then associated with a particular \textit{label} with semantic meaning \citep{Steinberg2015128}. For example, imageries of 'coral' regions receive the label 'coral'. Unfortunately, while bathymetric data can often be measured with decent accuracy at a distance (for example, with SONAR from ships at sea level), such visual imagery can only be obtained through expensive AUV missions that travel deep into the ocean to image underwater environments at a close distance. Together with the immense spatial scale of the benthic seafloor to be explored, this implies that it is impractical to map exhaustively the entire region of interest (ROI) under any reasonable time and cost. Furthermore, AUV missions are limited by power supply, data storage, and computational capabilities \citep{AsherBender}, further limiting the time and hence coverage each AUV mission can achieve. 
		
		As such, AUVs must prioritise exploring sub-domains of the ROI that ideally contain the most important and valuable information. This is a form of spatial sampling problem \citep{Rigby:ROB20372}, which aims to address the question: given the choice to observe only a few parts of the region of interest, how should one infer the best candidates for observation? AUV missions add another layer of complication to the spatial sampling problem - the candidate locations must form continuous paths that the AUV can physically travel.
		
		 This is known as the \textit{informative path planning problem}. The objective of informative path planning is to minimise the overall uncertainty regarding the entire region of interest.
		
		This thesis addresses the informative path planning problem for benthic habitat mapping. There are two main aspects to informative seafloor exploration for which this thesis is concerned with. The first part of this thesis is focused on benthic habitat mapping, where techniques of habitat classification and inference are discussed. The basic theory under which information and uncertainty are measured and quantified are developed and formulated in the general framework, which is then applied for benthic habitat mapping. The second part of this thesis then proceed to investigate the informative seafloor exploration problem. Using the properties of inference models developed for benthic habitat mapping, a range of path planning policies are discussed and compared. Practical considerations of computational tractability and flexibility then leads to methods the compromises between optimality and feasibility. Finally, this thesis proposes a practical framework for AUVs to autonomously plan informative paths that achieves the highest mapping rate under a classification accuracy criterion.
		
%	We propose linearised differential entropy (LDE) of Gaussian process classifiers (GPC) as the acquisition objective. We demonstrate through derivation that the LDE approximates an appropriate form of mutual information through taking advantage of the structure of GP classifiers and their likelihood responses.
%	
%	Finally, we evaluate the receding horizon approach under various acquisition criterion with collected datasets from Scott Reef \citep{IMOS}. We demonstrate the advantages of such an approach over simpler methods such as greedy and open loop methods, as well as computationally expensive methods such as acquisition over of joint information entropy estimated from Monte Carlo sampling.  Under a receding horizon formulation, we demonstrate that LDE acquisition achieves highest rate of information acquisition with respect to the misclassification criterion.
%	
%	The remainder of this paper is structured as follows. Section \ref{Section:RelatedWork} introduces related work in informative seafloor exploration. Section \ref{Section:LinearisedEntropy} introduces and derives the linearised differential entropy (LDE) of Gaussian process classifiers (GPC). Section \ref{Section:RecedingHorizonFormulation} motivates the general receding horizon approach in its basic form. Section \ref{Section:ExperimentalResults} demonstrates the receding horizon approach under LDE acquisition with experimental results on the Scott Reef dataset. Finally, section \ref{Section:Conclusion} provides concluding remarks.
%	
%		This thesis investigates machine learning methods for autonomous underwater vehicles to intelligently and actively plan an underwater path for data collection. The problem is complicated by the presence of dynamic uncertainties that is highly dependent on the vehicle's planning actions. A dynamic planning method is proposed to maximises information, or entropy, gained in a given underwater region, while constrained by cost and time.
%	
%		Intuitively, this information gathering problem, sometimes referred to as \textit{active sampling}, is dynamic in the sense that observing one part of the ocean modifies uncertainties about other parts of the ocean. The aim is that, with a well designed path planner, the final trajectory can maximise information gained about the habitats and resources residing in a large part of the ocean.
		
\section{Objectives}

	The high level objective of this thesis is to develop an informative seafloor exploration policy that can produce benthic habitat maps efficiently. Specifically, this thesis focuses on the theoretical and computational aspects of informative path planning that is practical for seafloor mapping. The aim is to address the informative seafloor exploration problem in a principled manner with theoretical grounding, while taking computational feasibility into account. 
	
	The goal of this thesis can be summarised as follows:
	
	\begin{quote}
		To investigate informative seafloor exploration policies for an AUV with limited mission time, in order to map benthic habitats faster in a principled and computationally feasible way.
	\end{quote}
%	
%		The high level objective of this thesis is to develop and design an underwater path planner such that the resulting path minimises the overall mapping error of the seafloor region of interest.
%		
%		Detailed examination of this objective would raise details that would need to be made more specific.
%		
%		Firstly, the measure of entropy would be highly dependent on the quantities or qualities the vehicle is to search for, as well as the underwater region of interest. It would need to be examined to ensure that it is an appropriate measure of the environment uncertainty to be reduced that is relevant to the mission.
%		
%		Secondly, finding paths that subsequently minimises the overall entropy is fundamentally a dynamic programming and optimisation problem. As with any optimisation problem, problem constraints are to be defined and made clear. Within the presence of possibly difficult dynamical constraints, it is likely that simplifications are necessary at various stages of the thesis.
%		
%		Another consideration is the feasibility of the algorithm. While it can be difficult to measure the optimality of any solution proposed, it is often easier to examine the feasibility of the algorithm through studying the hardware constraints involved or the physical environment. One of the most important feasibility constraint relevant to this thesis is the computational capabilities of the vehicle computer hardware. Depending on the final proposed algorithm, it may not always be possible for the path planer to be executed in a fully online fashion. A more likely situation would involve planning a path to be executed for a duration of time and update the path through re-planning at a frequency much lower than the vehicle is navigating. For short missions, it may be more desirable to have the entire path planned offline.
%		
%		Other subtleties arise from the nature of the the underwater path planning problem. There is often no goal location, only an objective to maximise the information gained. This is a form of active path planning, which is similar to the active sampling philosophy but with dynamic constraints.
%		
%		Finally, other than the path planning algorithm, a major part of this thesis revolves around modeling the ocean environment accurately. Without an accurate understanding of the ocean environment, the vehicle cannot plan a path that is of reasonable significance. This thesis is to address the ways the ocean environment is to be modeled, develop these algorithms, and implement them on an ocean exploration setting.
%		
%		The detailed objective of this thesis is thus to examine and address all the concerns above, and to provide a better understanding towards the methods these active path planning problems can be approached.
		
	\newpage
	\section{Contribution}
	
		This thesis is concerned with mapping seafloor benthic habitats efficiently in a principled manner. Specific contributions of this thesis are:
		
		\begin{itemize}
		
			\item A computationally efficient, parallelisable implementation for multiclass Gaussian process (GP) classifiers. Time and spatial complexity of the classifier are further improved through taking advantage of the GP classifier structure under Laplace approximation.
			
			% While \textit{one versus all} (OVA) and \textit{all versus all} (AVA) multiclass classifiers have been employed in many parametric and non-Bayesian classifiers, this thesis proposes an OVA and AVA framework for non-parametric, Bayesian GP classifiers. In this framework, both the learning/training and prediction/inference stage are parallelisable. Furthermore, it avoids the computational complexity from Monte Carlo sampling that is commonly required in the inference stage.
			
			\item An alternative, analytically tractable entropy measure hereby named \textit{linearised differential entropy} (LDE). The linearised differential entropy of GP classifiers captures mutual information and uncertainty of a given region of interest. Specifically, instead of examining uncertainty with mainly prediction variance, linearised differential entropy also takes into account the prediction bias. Through addressing the bias-variance trade-off, a common challenge in machine learning algorithms, linearised differential entropy provides an alternative way to quantify uncertainty and information. Analytical tractability is then achieved through linearisation approximations. Linearised differential entropy is proposed as an acquisition function for informative path planning, hereby referred to as \textit{LDE acquisition}.
			
			\item A receding horizon framework to informative path planning. This aims to provide a framework that is computationally efficient yet produce informative paths that are stable and near optimal. Together with LDE acquisition, the suitability of the receding horizon approach is demonstrated on the Scott Reef data set from \cite{IMOS}. This provides a computationally efficient approach to path planning that is able to run on an AUV in an online manner with suitable hardware. This thesis demonstrates that under a receding horizon framework, acquisition under linearised differential entropy achieves a faster mapping rate than traditional Monte Carlo approaches, and does so with less computational time. 
			
		\end{itemize}
		
	\newpage
	\section{Structure}
	
		The structure of this thesis is outlined as below.
		
%		Section \ref{Section:RelatedWork} introduces related work in informative seafloor exploration. Section \ref{Section:LinearisedEntropy} introduces and derives the linearised differential entropy (LDE) of Gaussian process classifiers (GPC). Section \ref{Section:RecedingHorizonFormulation} motivates the general receding horizon approach in its basic form. Section \ref{Section:ExperimentalResults} demonstrates the receding horizon approach under LDE acquisition with experimental results on the Scott Reef dataset. Finally, section \ref{Section:Conclusion} provides concluding remarks.
		
		Chapter \ref{Background} provides the necessary background and theory for the purpose of understanding this thesis. Related work in informative seafloor exploration are presented, as well as the various approaches that has been undertaken in this area. Most importantly, this chapter introduces and summarises the Gaussian process framework for both regression and classification. Basic concepts involved in active sampling and informative path planning are then discussed. 
		
		Chapter \ref{Benthic-Habitat-Mapping} details how the benthic habitat environment are modeled upon bathymetric features using Gaussian process classifiers. A general OVA and AVA framework for multiclass GP classification is developed, which is then applied to test and real datasets to verify performance. The highlight of this chapter is the introduction and derivation of the linearised differential entropy (LDE) for both binary and multiclass GP classifiers. Comparisons with the usual prediction information entropy is presented to demonstrate their differences and respective advantages and disadvantages. This motivates the use of linearised differential entropy as a suitable acquisition function for informative path planning.
		
		Chapter \ref{Informative-Seafloor-Exploration} formulates the receding horizon approach to informative path planning, and demonstrate the application of LDE acquisition under this formulation. Description of the simulation experiment are provided, and properties of the approach observed from corresponding results are discussed and analysed. The performance of the proposed exploration policy are then assessed with a classification accuracy criterion.
		
		Finally, chapter \ref{Conclusion} summarises the work presented in this thesis, the contributions made, and potentials for future improvements.
		
%		This thesis will rely heavily on the ocean environment modeling formulation presented in \cref{Background:OceanEnvironmentModeling}. The proposed methodology for environment modeling involves the use of Gaussian process models, whose background material is singled out in \cref{Background:GaussianProcesses} for further detailed coverage. Background knowledge for the path planning aspect of this thesis is then summarised in \cref{Background:PathPlanning}.
%				
%		Chapter {\color{BurntOrange} 3} \footnote{ {\color{BurntOrange} The following chapters are the contents currently proposed to be included in the final thesis, and is therefore subjected to change. The chapter 3 discussed here is not the same as the progress report currently included in \cref{ProgressReport}, although the material in the progress report ultimately be expanded and included in these later chapters.} } presents the implemented method for environment modeling using Gaussian processes. Simulations results and and tests are discussed to assess the performance of these models on bathymetric modeling and environment label prediction. Regression models and classification models are discussed in section {\color{BurntOrange} 3.2} and {\color{BurntOrange} 3.3} respectively.
%		
%		Chapter {\color{BurntOrange} 4} demonstrates the path planning process with partially observable Markov decision processes (POMDPs) in conjunction with the Gaussian process models. Important comparisons with simplified methods combining probabilistic road maps (PRM) and $A^{\star}$ search algorithm will motivate the use of Markov decision processes in section {\color{BurntOrange} 4.2}. The background material from \cref{Background:PathPlanning} will be expanded in section {\color{BurntOrange} 4.3} through a detailed discussion of Markov decision processes in the ocean exploration setting. Observability limitations would then invoke the POMDP formulation whose implementation will also be carefully outlined in section {\color{BurntOrange} 4.4}.
%		
%		Chapter {\color{BurntOrange} 5} focuses on the implementation of the above algorithms onto test hardware \footnote{ {\color{BurntOrange} This will depend strongly on the final scope of the thesis.} }. Comparisons of experimental results to simulation results are shown in section {\color{BurntOrange} 5.2}. A simplified and computationally cheaper alternative to the proposed methodology in Chapter {\color{BurntOrange} 4} is outlined in section {\color{BurntOrange} 5.3}. Final results and pseudocode are shown in section {\color{BurntOrange} 5.4}.
%		
%		Chapter {\color{BurntOrange} 6} summarises the work presented in this thesis, the contributions made, and potentials for future improvements.
%		
%		Finally, the appendices support the thesis with useful information that is not immediately relevant to the main work presented.
		
		