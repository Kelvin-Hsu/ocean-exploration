\chapter{Conclusion}
\lhead{Conclusion}
\label{Conclusion}

	This thesis investigates and develops informative seafloor exploration methods for benthic habitat mapping. Through concluding the work in informative path planning using linearised model differential entropy acquisition under receding horizon formulation, this section discusses the future outlook of this thesis work.
	
	\section{Contributions and Significance}
		
		Of the contributions listed in Section \ref{Introduction:Contribution}, the development and implementation of a new acquisition criterion, the linearised model differential entropy, and an informative path planning framework, the receding horizon formulation, are two of the most vital contributions of this thesis. Together, they form a complete informative path planning technique for benthic habitat mapping which is stable and relatively efficient compared to other methods. In effect, it allows a non-myopic, mutual informative exploration scheme that outperforms both Monte Carlo methods and other approaches without such properties. With a novel, principled, and tractable informative exploration policy that outperforms existing solutions, this concludes the mission of this thesis.
		
		Another major significance and implication of this work is its feasibility, tractability, and flexibility. Through developing a simple, efficient, and parallelisable multiclass classifier in the OVA and AVA scheme, coupled by improved probability fusion techniques, such a GP classification framework can be readily incorporated to other applications or extended further in the seafloor mapping domain.
		
%		Firstly, it avoids the need to compute acquisition criterion throughout the seafloor region, which is extremely costly in time for non-mutual information measures such as AMPIE, and furthurmore memory intensive for mutual information measures such as MCPIE and LMDE. As a result, this formulation make feasible the use of mutual information measures. Secondly, it allows a feedback structure to path planning and avoids executing outdated path plans. Thirdly, with an appropriate horizon structure, the method provides a balance between non-myopic planning and immediate feasibility.
		
%		Finally, these acquisition criteria and informative path planning schemes are evaluated in performance with the Scott Reef dataset, which reveals the effectiveness of MCPIE and especially LMDE acquisition approaches over non-mutual, myopic, and non-informative methods (Section \ref{InformativeSeafloorExploration:ScottReef}). 
	
	\section{Future Work and Implications}
	
		The work presented in this thesis calls for further work in improving the proposed informative path planning techniques and considering its implications.
		
		One of the most intriguing concept derived from this thesis is the relationship between prediction information entropy and model differential entropy. Semantic interpretations of their meanings have suggested that they are correspondingly similar to variance and bias, which are two sides of the ``coin'' of uncertainty. Further statistical analysis and investigation with probability theory could be interesting in revealing how these measures can be combined for a mutual information measure that allows the agent to intelligently discern uncertainty by variance and uncertainty by bias. With a hopeful outlook such a measure may be able to enable better acquisition properties for mapping problems that requires classification.
		
		On top of more investigations in the acquisition criteria, the path planning scheme itself can also be improved. The work of this thesis follows the common philosophy of transcribing a path planning problem in the continuous domain to the discrete domain. While this is a common and perfectly valid approach, it is certainty superseded by techniques that allow planning directly in the original continuous domain, in order to avoid discretisation approximations and hence limitations. 
		
		Nevertheless, there are also immediate work that can be investigated which builds on the current discretised formulation. Firstly, the horizon length could be reduced as the agent reaches the end of each mission. As no missions will continue indefinitely, it may be beneficial to let the agent know that its mission will terminate shortly such that it will switch to a more greedy approach before its time is up. More importantly, however, are the choices of the mission length and stopping criteria. Currently, the missions lengths are chosen based on experiences with past missions and the power limitations of the AUV. Nevertheless, it can be interesting to investigate the case where the AUV itself decides when to stop a mission, surface to sea level, and have a ship deliver it to a new location where it finds the most informative. This is based on the fact that underwater velocities are much slower than surface velocities. Such a scheme would involve the AUV constantly evaluating its surroundings against faraway regions for potential information. With appropriately devised loss functions, the AUV would decide whether it has explored the current area enough such that it is time to move to a new area.
			
		The tractability and accuracy of the methods developed above can also be improved. In terms of tractability, one of the major limitations of this work is that the datasets are large in quantity. Much work has been done in the machine learning community for efficient approximations of Gaussian process inference, such as sparse Gaussian process approximations where weak kernel covariances are discarded to save both memory and computational time. In terms of accuracy, while Laplace approximation achieves good performance, there are other approximation techniques that produce even more accurate results, although unfortunately at the expense of higher time complexity. Examples of such approximation methods are Expectation Propagation and Variance Inference \citep{GaussianProcessForMachineLearning}. The advantage of the work contributed in this thesis is that it is compatible with all such improvement techniques described in the literature.
		
		The work presented in this thesis is in fact not limited to the underwater domain. It can be used in any setting which requires building a map of discretely labeled objects. Typical examples include mapping the object occupancy of an area, aerial monitoring of agricultural lands with unmanned aerial vehicles, and terrain feature recognition with planetary exploration rovers. As such, the methods presented in this thesis could be tested on these scenarios to better evaluate performance, and further serve to inspire ways of improvements.
		
		The above application domains also motivate the need for a spatial-temporal model. In benthic habitat mapping, it is assumed that the true habitat map stay constant throughout the mapping procedure, so that observations made a long time ago are equally as valid as observations made recently. However, environmental monitoring problems such as object occupancy mapping of an dynamic environment involve a dynamically changing ground truth. While spatial-temporal models based on Gaussian processes have been investigated before \citep{Roman:SequentialBayesianOptimisation}, these methods typically only address the mapping of continuous targets which are regression problems. Further work building upon the Gaussian process classification scheme developed here is thus an interesting area to explore.
		
		Nevertheless, this thesis has presented a computationally viable approach to informative path planning through a novel acquisition function, the linearised model differential entropy. Under a receding horizon formulation, linearised model differential entropy acquisition has demonstrated significant improvement from traditional methods in the case of Scott Reef, which concludes the mission of this thesis.

	