\chapter{Conclusion and Future Work}
\lhead{Conclusion and Future Work}
\label{Conclusion}

	This thesis investigates and develops informative seafloor exploration methods for benthic habitat mapping.
	
	After an introduction to the theory behind Bayesian modeling with Gaussian processes (Section \ref{Background:GaussianProcesses}), a review of current literature in informative path planning quickly reveals the need for a more computationally efficient approach that is non-myopic while capturing mutual information (Section \ref{Background:RelatedWork}).
	
	The outset of this thesis then begins by building upon the existing Gaussian process classification framework for benthic habitat mapping (Section \ref{BenthicHabitatMapping}). The benthic habitat mapping problem is first formulated as a supervised learning problem which requires the habitat labels to be modeled upon appropriate bathymetric features through classification. This requires an introduction to bathymetric features (Section \ref{BenthicHabitatMapping:BathymetricFeatures}). Through examining the usual binary classification framework, efficient and parallelisable multiclass classifiers are developed. The All versus All GP classifier is formulated in detail which parallels the way One versus All GP classifiers have been used in. Furthermore, on top of simple normalisation, two novel probability fusion methods, mode keeping and exclusion, are introduced which exhibit better inference properties with respect to computing the prediction information entropy (Section \ref{BenthicHabitatMapping:Classification}). The benthic habitats across the Scott Reef seafloor are then mapped with such classifiers which also provides the initial scenario for path planning to be used later on (Section \ref{BenthicHabitatMapping:ScottReef}).
	
	The work then moves on to investigate informative path planning methods that build upon such Gaussian process classifiers (Section \ref{InformativeSeafloorExploration}). Through motivating the need for a measure of mutual information, a direct approach with Monte Carlo estimation is formulated in order to compute the \textit{joint} prediction information entropy, hereby named Monte Carlo prediction information entropy (MCPIE) (Section \ref{InformativeSeafloorExploration:MCPIE}). Careful analysis of the predictive probabilities obtained under likelihood response transformations of the latent Gaussian process then inspires the concept of model differential entropy (MDE) of a Gaussian process classifier. Analytical tractability is then achieved through a linearisation technique for both the binary, multiclass OVA, and multiclass AVA scenario, thus producing the linearised model differential entropy (LMDE) mutual information measure (Section \ref{InformativeSeafloorExploration:LMDE}). The two measures are then compared in properties and performance, which reveals the complementary nature of the two mutual information measures that resembles the bias-variance trade-off that is ubiquitous in statistics and machine learning (Section \ref{InformativeSeafloorExploration:ComparisonMutualEntropyMeasures}).
	
	A stable and efficient path planning scheme is then developed under the receding horizon formulation (Section \ref{InformativeSeafloorExploration:RecedingHorizonFormulation}). While only optimal in finite horizon at each time instance, such an approach exhibits the following advantages compared to traditional informative path planning methods. Firstly, it avoids the need to compute acquisition criterion throughout the seafloor region, which is extremely costly in time for non-mutual information measures such as AMPIE, and furthurmore memory intensive for mutual information measures such as MCPIE and LMDE. As a result, this formulation make feasible the use of mutual information measures. Secondly, it allows a feedback structure to path planning and avoids executing outdated path plans. Thirdly, with an appropriate horizon structure, the method provides a balance between non-myopic planning and immediate feasibility.
	
	Finally, these acquisition criteria and informative path planning schemes are evaluated in performance with the Scott Reef dataset, which reveals the effectiveness of MCPIE and especially LMDE acquisition approaches over the AMPIE and GREEDY-PIE approaches, as well as non-informative methods (Section \ref{InformativeSeafloorExploration:ScottReef}).
	
	The work presented in this thesis calls for further work in improving and better evaluating the informative path planning methods proposed.
	
	One of the most important and intriguing is the relationship between prediction information entropy and model differential entropy. Semantic interpretations of their meanings have suggested that they are correspondingly similar to variance and bias, which are two sides of the ``coin'' of uncertainty. Further statistical analysis and investigation with probability theory could be interesting in revealing how these measures can be combined for a mutual information measure that allows the agent to intelligently discern uncertainty by variance and uncertainty by bias. With a hopeful outlook such a measure may be able to enable better acquisition properties for mapping problems that requires classification.
	
	On top of more investigations in the acquisition criteria, the path planning scheme itself can also be improved. The work of this thesis also follows the philosophy of transcribing a path planning problem in the continuous domain to the discrete domain. While this is a common approach, it is certainty superseded by techniques that allow planning directly in the original continuous domain, in order to avoid discretisation approximations and hence limitations. 
	
	Nevertheless, there are also immediate work that can be investigated which builds on the current discretised formulation. Firstly, the horizon length could be reduced as the agent reaches the end of each mission. As no missions will continue indefinitely, it may be beneficial to let the agent know that its mission will terminate shortly such that it will switch to a more greedy approach before its time is up. More important however are the choices of the mission length and stopping criteria. Currently, the missions lengths are chosen based on experiences with past missions and the power limitations of the AUV. Nevertheless, it can be interesting to investigate the case where the AUV itself decides when to stop a mission, surface to sea level, and have a ship deliver it to a new location where it finds the most informative. This is based on the fact that underwater velocities are much slower than surface velocities. Such a scheme would involve the AUV constantly evaluating its surroundings for potential information. With appropriately devised loss functions, the AUV would decide whether it has explored the current area enough such that it is time to move to a new area.
		
	The tractability and accuracy of the methods developed above can also be improved. In terms of tractability, one of the major limitations of this work is that the datasets are large in quantity. Much work has been done in the machine learning community for efficient approximations of Gaussian process inference, such as sparse Gaussian process approximations where weak kernel covariances are discarded to save both memory and computational time. In terms of accuracy, while Laplace approximation achieves good performance, there are other approximation techniques that produce even more accurate results, although unfortunately at the expense of higher time complexity. Examples of such approximation methods are Expectation Propagation and Variance Inference \cite{GaussianProcessForMachineLearning}.
	
	The work presented in this thesis is in fact not limited to the underwater domain. It can be used in any setting which requires building a map of discretely labeled objects. Typical examples are mapping the occupancy of an area, classifying between the objects such as cars and buildings, using an aerial monitoring system such as an unmanned aerial vehicle. Another application could be an autonomous ground vehicle building a map of the types of vegetation or environment distributed across an area. As such, the methods presented in this thesis could be tested on these scenarios to better evaluate performance, which can also serve to inspire ways of improvements as above.
	
	The above application domains also motivates the need for a spatial-temporal model. In the benthic habitat mapping scenario, it is assumed that the habitat map stay constant throughout the mapping procedure, so that observations made a long time ago are equally as valid as observations made recently. However, environmental monitoring problems such as simple occupancy mapping at a dynamic environment involve a dynamically changing ground truth. While spatial-temporal models based on Gaussian processes have been investigated before \cite{Roman:SequentialBayesianOptimisation}, these methods typically only address the mapping of continuous outputs which are regression type problems. Further work building upon the Gaussian process classification scheme developed is thus an interesting area to explore.
	
	

	