\lhead{Modeling the Ocean Environment}
\chapter{Modeling the Ocean Environment}
\label{ModelingOceanEnvironment}

%	\section{Gaussian Process Classification with Laplace Approximation}
%	
%%		\subsection{Theory}
%%		
%%		\subsection{Implementation}
%%		
%%		\subsection{Results}
%		
%	\section{Gaussian Process Classification with Probabilistic Least Squares}
%
%%		\subsection{Theory}
%%		
%%		\subsection{Implementation}
%%			
%%		\subsection{Results}
%		
%	\section{Non Stationary GP Classification}
%	
%%		\subsection{Theory}
%%		
%%		\subsection{Implementation}
%%			
%%		\subsection{Results}
%		
%	\section{Drawing from Gaussian Process Classifiers}
%				
%%		\subsection{Theory}
%%		
%%		\subsection{Implementation}
%%			
%%		\subsection{Results}
				
	\section{Modeling on Test Data Sets}
	
	\section{Comparison: Laplace Approximation and Probabilistic Least Squares}
	
		Compare probability outputs, classification outputs, entropy outputs, etc
	
	\section{Comparison: OVA and AVA methods for multiclass classifiers}
	
	\section{Comparison: Probability fusion methods for multiclass classifiers}
	
%		\subsection{Normalisation Method}
%			
%		\subsection{Mode Keeping}
%			
%		\subsection{Exclusion}
			
	\section{Modeling the Scott Reef Environment}
	
		\subsection{Problems and Solutions to Big Data Analysis}
		
		\subsection{Feature Extraction}
		
		\subsection{Case with 4 Labels}
		
			(With different amounts of sampled points)
			
		\subsection{Case with 17 Labels}
		
	\section{Measuring Mutual Information}
	
		\subsection{Motivation}
			For both modeling purposes and path planning purposes, simply knowing the entropy at each query point is not enough. 
			
		\subsection{Shannon Entropy}
		
		\subsection{Lack of a Closed Form Solution}
		
	\section{A Direct Approach: Monte Carlo Sampling}
		
		\subsection{Development of Methodology}
		
			Provide pseudocode for limited and good way of doing it
			
		\subsection{Binary Classification}
		
		\subsection{Multi-class Classification}
		
	\section{A Faster Approach: Linearised Entropy}
		
		\subsection{Development of Methodology}
		
		\subsection{Binary Classification}
		
			For binary classification, linearisation is performed on the sigmoid, or response, function.
		
			The queried latent vector $\vec{f}$, a finite collection of latent function instances at query points, are distributed as a multivariate Gaussian distribution which can be computed from (equation). 
			
			\begin{equation}
				\vec{f} = [f_{1}, f_{2}, \dots, f_{n_{q}}]^{T} \sim \mathcal{N}(\vec{\mu}, \Sigma)
			\end{equation}
				
			\begin{equation}
				\pi_{i} = \sigma(f_{i}) \qquad \qquad \forall i \in I_{\mathrm{query}} = {1, 2, \dots, n_{q}}
			\end{equation}
			
		\subsection{Multi-class Classification}
		
			For multi-class classification, linearisation is performed on the softmax function for each class.
			
			Provide intuitive reason for the squashing.
			
		